%! Author = chicken1925
%! Date = 2020/12/19

% Preamble
\documentclass[11pt]{jsarticle}

% Packages
\usepackage{amsmath}
\usepackage{amsfonts}
% Document
\begin{document}
2119116s 佐野 海徳\\
HW49\\
\\HW50
\par $L = \mathbb{Q}(\sqrt{2},\sqrt{3})/\mathbb{Q}$の次数は4であり、$\mathbb{Q}(\sqrt{2},\sqrt{3}))/\mathbb{Q}(\sqrt{2}), \mathbb{Q}(\sqrt{2},\sqrt{3})/\mathbb(\sqrt{3})$はそれぞれ次数が2次、
$\mathbb{Q}(\sqrt{2})/\mathbb{Q},\mathbb{Q}(\sqrt{3})/\mathbb{Q}$はそれぞれ次数が2である。
ここで、Galois群は位数4であるから$c_4$か$c_2 \times c_2$と同型。この中で位数2の部分群を2つ持つのは$c_2 \times c_2$。$\mathbb{Q}(\sqrt{2})$に$\{(0,0),(1,0)\}$,$H^{\prime} = \mathbb{Q}(\sqrt{3})$に$H^{\prime\prime}\{(0,0),(0,1)\}$を
対応させる。このとき、$\mathbb{Q}(\sqrt{2}) = L^{H^{\prime}}, \mathbb{Q}(sqrt{3}) = L^{H^{\prime\prime}}$と対応するので、Lの中間体は$L,\mathbb{Q}, \mathbb{Q}(\sqrt{2}), \mathbb{Q}(\sqrt{3})$の4つである。



\end{document}