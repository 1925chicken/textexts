%! Author = chicken1925
%! Date = 2020/12/14

% Preamble
\documentclass[11pt]{jsarticle}

% Packages
\usepackage{amsmath}
\usepackage{amsfonts}
% Document
\begin{document}
2119116s 佐野 海徳\\
HW47
\par$64k + 1$という形をもつ数を下から探索していくと、641で$F_5$が割り切れることが分かる。
これを改めて試し割りするとこれは素数であるので641が$F_5$の素因数の1つ。\\
HW48
\par $x^5 + x^4 + x^3 - 2x^2 - 2x - 2 = 0$は因数分解すると$(x^2 + x + 1)(x^3 - 2) = 0$となり、左辺の因数となっている式はそれぞれ複素数の範囲で一次式に分解できる。
それらの根を$\mathbb{Q}$に添加し、その体をLとすると$L/ \mathbb{Q}$はべき根による拡大であり、$x^5 + x^4 + x^3 -2x^2 - 2x - 2$はL係数と見れば1次式の積に分解できるためこれが具体例である。

\end{document}