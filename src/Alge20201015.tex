%! Author = chicken1925
%! Date = 2020/10/15

% Preamble
\documentclass[11pt]{jsarticle}

% Packages
\usepackage{amsmath}

% Document
\begin{document}
2119116s 佐野 海徳
HW19 \\
まず、各元の位数は、${e,(12)(34),(13)(24),(14)(23)} = {1,2,2,2}$である。なぜなら、単位元以外は同じ数を含まない置換であり、なおかつそれぞれの元を構成する置換の位数が2であることから、それらを合成した置換の位数は2と2の最小公倍数である2であるといえる。
更に、単位元以外の元の位数が2である群はアーベル群である。a,bを群Gの元とすると$(ab)(ab) = e, aa = e,bb = e$がいえる。すなわち$(ab) = (ab)^{-1},a = a^{-1},b = b^{-1}$であるからである。また一般に$(ab)^{-1} = b^{-1}a^{-1}$であり、$ab = (ab)^{-1} = b^{-1}a^{-1} = ab$だから、ab = baが成り立つためGはアーベル群である。
また、また位数4で直積で表されるアーベル群は$C_2 \times C_2$のみであるからこの群はこの$S_4$の部分群は$C_2 \times C_2$と同型である。

HW20\\
eの位数が1であるのは自明。まず、$\sigma$は$\frac{2\pi}{4} = \frac{\pi}{2}$回転する操作であるので$\sigma^2$は$\pi$だけ回転する操作。$\sigma^3$は$\frac{3\pi}{2}$回転する操作、$\sigma^4$は$2\pi$だけ回転する操作であり、結果的にeと等しい。また、この操作で1と番号を振った頂点がもう一度1に戻るのは
$2n\pi$回転(nは任意の整数)したとき、言い換えればべきが4の倍数になるときである。よって、$\sigma$の位数は4,$\sigma^2$の位数は2,$\sigma^3$の位数は4。\\
次に$\tau$について考える。$\tau$はもう一度自身と同じ操作をすればeになるので位数は2。$\tau\sigma$の位数を考える。$\tau$の位数は2,$\sigma$の位数は4であることから最小公倍数を取って$\tau\sigma$の位数は4。同様にして$\tau\sigma^2$の位数は2、$\tau\sigma^3$の位数は6である。
\end{document}