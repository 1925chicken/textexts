%! Author = chicken1925
%! Date = 2020/11/26

% Preamble
\documentclass[11pt]{article}

% Packages
\usepackage{amsmath}
\usepackage{amsfonts}

% Document
\begin{document}
2119116s 佐野海徳\\
HW37
\par 2が既約でないならa,b,c,d$\in \mathbb{Z}$を用い、共役をとると$2 = (a + \sqrt{-5}b)(c + \sqrt{-5}d),2 = (a - \sqrt{-5}b)(c - \sqrt{-5}d)$と書ける。
辺々掛けると$4 = (a^2 + 5b^2)(c^2 + 5d^2)$となるが、この等式を満たす整数の組は(a,b,c,d) = (2,0,2,0)以外に存在しない。つまり、どう書き表しても2 = 2 + 0i,つまり2であるから、2は既約元。
同様の条件で3について議論すると、3が既約でないなら$3 = (a + \sqrt{-5}b)(c + \sqrt{-5}d),3 = (a - \sqrt{-5}b)(c - \sqrt{-5}d)$と表されるはずであるが、辺々掛けると
$9 = (a^2 + 5b^2)(c^2 + 5d^2)$となるが、これを満たす整数の組は(a,b,c,d) = (3,0,3,0)以外は存在しない、つまり3としか表せないのでこれも既約。
$1 + \sqrt{-5} = (a + \sqrt{-5}b)(c - \sqrt{-5}d), 1 - \sqrt{-5} = (a - \sqrt{-5}b)(c - \sqrt{-5}d)$と表せるとするなら、辺々掛けて$6 = (a^2 + 5b^2)(c^2 + 5d^2)$となるが、
これを満たすのは(a,b,c,d) = $(\pm1,\pm1,\pm1\,\pm1)$であり、この中に$1 \pm \sqrt{-5}$となる組み合わせが含まれている。よって$1\pm \sqrt{-5}$も既約元である。\\
HW38
\par 4次正方行列の異なる固有値をそれぞれ$\alpha,\beta,\gamma,\sigma$とおく。4を自然数の組み合わせで分割すると、
$4 = 1 + 1 + 2 = 1 + 3 = 1 + 1 + 1 + 1 = 2 + 2$となる。よって4次正方行列におけるジョルダン標準形は\\
$\left(
\begin{array}{cccc}
    \alpha & 1 & 0 & 0 \\
    0 & \alpha & 1 & 0 \\
    0 & 0 & \alpha & 1 \\
    0 & 0 & 0 & \alpha
\end{array}
\right),
\left(
\begin{array}{cccc}
    \alpha & 0 & 0 & 0 \\
    0 & \beta & 0 & 0 \\
    0 & 0 & \gamma & 1 \\
    0 & 0 & 0 & \gamma
\end{array}
\right),\\
\left(
\begin{array}{cccc}
    \alpha & 0 & 0 & 0 \\
    0 & \beta & 1 & 0 \\
    0 & 0 & \beta & 1 \\
    0 & 0 & 0 & \beta
\end{array}
\right),
\left(
\begin{array}{cccc}
    \alpha & 0 & 0 & 0 \\
    0 & \beta & 0 & 0 \\
    0 & 0 & \gamma & 0 \\
    0 & 0 & 0 & \sigma
\end{array}
\right),\\
\left(
\begin{array}{cccc}
    \alpha & 1 & 0 & 0 \\
    0 & \alpha & 0 & 0 \\
    0 & 0 & \beta & 1 \\
    0 & 0 & 0 & \beta
\end{array}
\right)
$の5種類である。


\end{document}