%! Author = chicken1925
%! Date = 2020/11/13

% Preamble
\documentclass[11pt]{jsarticle}

% Packages
\usepackage{amsmath}
\usepackage{amssymb}
% Document
\begin{document}
2119116s 佐野 海徳\\
HW31
\par $T \geq 2$とする。
\par (1)イデアルのうち、Tから2つ以上の元を除いた集合の冪集合で構成されるもの。(ex. T = \{1,2,3\}なら\{1\},\{2\},\{3\},$\{\phi\}$ それぞれの冪集合を例として考えれば良い。)取り除いた元をa,b,c....とすると、
この中の任意の2つの元のみを要素に持つ部分集合同士の積(ex.\{a\}\{b\})で$\phi$が計算されるが、このとき、そのイデアルをIとして$\{a\} \notin I,\{b\} \notin I$が仮定より明らかである。よって定義を満たさず素イデアルではない。
\par (2)Sからひとつだけ元を除いたものの冪集合で構成されるイデアルI。除いた元をxとし、xとその他の任意の元を含む集合の冪集合をX,その元のうち、
xを元として含む集合の集合の元をY(ex. T = \{1,2,3\}で3をのぞいたなら、3を元にもつ集合の集合は\{\{1,3\},\{2,3\},\{1,2,3\},\{3\}\}であるからYの候補は\{1,3\},\{2,3\},\{1,2,3\},\{3\}である。)とし、
イデアル内の任意の元をVとすると、$ YV  = \phi$となるのは、Vがxを元として持つ集合でありえないことから明らかである。また、これは仮定より$V \in I, \phi \in I$であるから素イデアルの定義を満たす。
よって、T:finite set pe(T)のイデアルのうち、素イデアルとなるのは$\mathfrak{P}(T - x)$(xはTの任意の元1つ)に限られる。
\\
HW32
\par
$84 = 2^2 \times 3 \times 7$であるから、$\phi(84) = \phi(2^2) \times \phi(3) \times \phi(7) = (2^2 - 2^1) \times (3^1 - 3^0) \times (7^1 - 7^0) = 2 \times 2 \times 6 = 24$である。
よって(Z/84Z)\^xの位数は24。

\end{document}