%! Author = chicken1925
%! Date = 2020/11/18

% Preamble
\documentclass[11pt]{jsarticle}

% Packages
\usepackage{amsmath}
\usepackage{amsfonts}
% Document
\begin{document}
    2119116s 佐野海徳\\
    \fbox{1}
    \par (1)のベクトルを横に並べて$A = (a_1,a_2,a_3,a_4,a_5)$として簡約化すると、\\
    $\left(
    \begin{array}{ccccc}
        1 & 2 & 1  & 1  & 3  \\
        0 & 1 & 0  & -1 & 2  \\
        1 & 0 & -1 & 2  & 3  \\
        1 & 1 & 2  & 4  & -1 \\
    \end{array}
    \right)$ $\to
    \left(
    \begin{array}{ccccc}
        0 & 1 & 1  & 0  & 0 \\
        0 & 1 & 0  & -1 & 2 \\
        1 & 0 & -1 & 2  & 3 \\
        0 & 0 & 2  & 2  & 4
    \end{array}
    \right)$\\
    $\to \left(
    \begin{array}{ccccc}
        0 & 1 & 1 & 0  & 0 \\
        0 & 1 & 0 & -1 & 2 \\
        1 & 0 & 0 & 3  & 1 \\
        0 & 0 & 1 & 1  & 2
    \end{array}
    \right) \to \left(
    \begin{array}{ccccc}
        1 & 0 & 0 & 3  & 1  \\
        0 & 1 & 0 & -1 & 2  \\
        0 & 0 & 1 & 1  & -2 \\
        0 & 0 & 0 & 0  & 0
    \end{array}
    \right)$\\
    となるから,$B = (b_1,b_2,b_3,b_4,b_5)$とおくと、$Ax = 0 \Leftrightarrow Bx = 0$であるから、$a_1,...a_5$の一次関係と$b_1,...b_5$の一次関係は等しい。
    よって、Bの列ベクトルから$b_1,b_2,b_3$が1次独立であることがわかり、r = 3,$a_1,a_2,a_3$が1次独立、$a_4 = 3a_1 - b_2 + b_3$,$a_5 = b_1 + 2b_2 - 2b_3$。\\
    \par(2)実際に計算すると$f_1 = f_3 + f_5, f_2 = f_4 + f_1, f_3 = f_1 - f_5, f_4 = f_2 - f_1,f_5 = f_1 - f_3$となるので1次独立なベクトルは存在しない。つまりr = 0。\\
    \fbox{2}
    (1)A = $\left(
    \begin{array}{ccccc}
        1 & 1  & 1 & 1  & 1 \\
        1 & -1 & 1 & 0  & 2 \\
        2 & 1  & 2 & -1 & 5
    \end{array}
    \right)$とおく。\\
    A = $\left(
    \begin{array}{ccccc}
        1 & 1  & 1 & 1  & 1 \\
        1 & -1 & 1 & 0  & 2 \\
        2 & 1  & 2 & -1 & 5
    \end{array}
    \right) \to \left(
    \begin{array}{ccccc}
        0 & 2  & 0 & 1  & -1 \\
        1 & -1 & 1 & 0  & 2  \\
        3 & 0  & 3 & -1 & 7
    \end{array}
    \right)\\
    \to \left(
    \begin{array}{ccccc}
        0 & 2  & 0 & 1 & -1 \\
        1 & -1 & 1 & 0 & 2  \\
        0 & 5  & 0 & 0 & 0
    \end{array}
    \right)$ $\to \left(
    \begin{array}{ccccc}
        1 & 0 & 1 & 0 & 2  \\
        0 & 1 & 0 & 0 & 0  \\
        0 & 0 & 0 & 1 & -1
    \end{array}
    \right)$となる。
    したがってAx = 0の解は$x_1 = c_1,x_5 = c_2$とおくと$x_1 = -c_1 - 2c_2, c_2 = 0, x_4 = c_2$。
    ゆえに$\left(
    \begin{array}{c}
        x_1 \\
        x_2 \\
        x_3 \\
        x_4 \\
        x_5
    \end{array}
    \right)$ $=
    \left(
    \begin{array}{c}
        -c_1-2c_2 \\
        0         \\
        c_1       \\
        c_2       \\
        c_2
    \end{array}
    \right)\\ =
    c_1 \left(
    \begin{array}{c}
        -1 \\
        0  \\
        1  \\
        0  \\
        0
    \end{array}
    \right)$ $=
    c_2 \left(
    \begin{array}{c}
        -2 \\
        0  \\
        0  \\
        1  \\
        1
    \end{array}
    \right)
    $
    となり、dim(W) = 2で、1組の基底として\{
    $\left(
    \begin{array}{c}
        1 \\
        0 \\
        1 \\
        0 \\
        0
    \end{array}
    \right)$
    $\left(
    \begin{array}{c}
        -2 \\
        0  \\
        0  \\
        1  \\
        1
    \end{array}
    \right)
    $
    \}がとれる。\\
    (2)A = $
    \left(
    \begin{array}{ccc}
        1 & 2  & -1 \\
        3 & -3 & 2
    \end{array}\right)
    $とする。
    簡約化すると、$
    \left(
    \begin{array}{ccc}
        9 & 0 & 1 \\
        5 & 1 & 0
    \end{array}
    \right)
    $となる。\\
    $x_2 = c_1,x_3,c_2$とすると、
    $\left(
    \begin{array}{c}
        x_1 \\
        x_2 \\
        x_3 \\
    \end{array}
    \right) = $ $
    \left( \begin{array}{c}
               -c_1 - 2c_2 \\
               c_1         \\
               c_2
    \end{array}
    \right) =
    $\\ $
    c_1 \left( \begin{array}{c}
                   9 \\
                   1 \\
                   0
    \end{array}
    \right) +
    $
    $c_2 \left(
    \begin{array}{c}
        5 \\
        0 \\
        1
    \end{array}
    \right)
    $\\
    となる。
    よってdim(W) = 2で1組の基底として\{
    $\left(
    \begin{array}{c}
        9 \\
        1 \\
        0
    \end{array}
    \right)$
    $\left(
    \begin{array}{c}
       5 \\
       0 \\
       1
    \end{array}
    \right)$
    \}が取れる。\\
    \fbox{3}
    A = $(f_1,f_2,f_3)$としAを簡約化する。\\
    $\left(
    \begin{array}{ccc}
        1 & -1 & 1\\
        -1 & 2 & 2\\
        1 & 2 & -1 \\
    \end{array}
    \right) =$
    $\left(
    \begin{array}{ccc}
        1 & -1 & 1 \\
        0 & 1 & 3 \\
        0 & 3 & 2
    \end{array}
    \right) = $\\
    $\left(
    \begin{array}{ccc}
        1 & -1 & 4 \\
        0 & 1 & 3 \\
        0 & 0 & 1
    \end{array}
    \right) = $
    $
    \left(
    \begin{array}{ccc}
        1 & 0 & 0 \\
        0 & 1 & 0 \\
        0 & 0 & 1
    \end{array}
    \right)
    $\\
    よって、この行列のrank= 3であるからAを構成する3本のベクトルは一次独立である。
    ゆえに$\{f_1,f_2,f_3\}$は$R[x]_2$の1組の基底である。
\end{document}