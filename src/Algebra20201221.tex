%! Author = chicken1925
%! Date = 2020/12/23

% Preamble
\documentclass[11pt]{article}

% Packages
\usepackage{amsmath}
\usepackage{amsfonts}
% Document
\begin{document}
2119116s 佐野海徳\\
HW51\\
巡回群Gの生成元をa,単位元を1で表す。Gの部分群をHと書く。Hの元はすべて生成元のべき$a^i(i \in \mathbb{Z})$で書ける。
a = 1,つまりGが単位元だけでなる群のときは自明。また、Hが単位元1のみからなるGの部分群であるとき、Hは巡回群。$a \neq 1$として
Hは単位元1の他にも元を持つとすると、$S = \{i \in \mathbb{Z} | i > 0 \& a^i \in H\}$となる集合Sを考える。Hについての仮定と$a^{-i} \in H \Leftrightarrow a^i \in H$
から$S \neq \emptyset$がわかる。したがってSは最小元mをもつ。このとき$H = <a^m>$が成り立つ。このとき$<a^m> \subseteq H$は明らか。逆に$a^n(n \in \mathbb{Z})$とする。
nをmで割ると$n = mq + r$となるある整数の組m,r$0 \leq r < m$が一意的に存在し、$a^r = a^n a^{-mq} \ in H$。mの最小性によりr = 0,故にn = mq。これより$a^n = a^{mq} \in <a^m>$。
よって$H \subseteq <a^m>$。


\end{document}