%! Author = 佐野海徳
%! Date = 2020/10/09

% Preamble
\documentclass[12pt]{jsarticle}

% Packages
\usepackage{amsmath}
\begin{document}
2119116s 佐野 海徳
\par 第一章ではリスクとは何か、リスクをどうやって評価・計量化するか、計量化して出てきたVaRの活用事例を説明している。まずリスクとは「組織の目標・目的達成に影響を与える事象の発生可能性」
として定義され、「影響の大きさと発生の可能性に基づいて測定される」とされる。また、リスクにはコントロール等の措置が全く整備されていない状態を仮定した場合に存在する「固有リスク」とコントロール等の措置を
講じた後に残る「残余リスク」がある。そして、「組織の目標・目的の達成に関して合理的補償を提供するため、発生する可能性のある事象や状況を識別、評価、管理、コントロールするプロセス」をリスクマネジメントという。
\par リスクを評価する方法として一般的に残余リスクの影響度が大きく発生可能性が高い方がより重要度が高い、つまりリスクマップで右上にある方が重要度がより高いとするリスクマップ方式、
残余リスクを評点化して評価するスコアリング方式、残余リスクの影響度を金額に換算して発生可能性、あるいは影響度が一定金額を超えたものを重要性が高いとするリスク軽量化方式がある。
この3つどれでもリスクの重要度や優先順位を決めることができるが、組織の収益・経営力と対比したリスクの大きさはリスク計量化方式でないと判定できない。リスク計量化方式には今すぐ手をうつべきリスクか、
時間をかけて対処していいのかが判断しやすいというメリットがある。
\par リスクの計量化とは、リスク事象の影響度を金額換算し、発生可能性を確率で表すことである。そしてリスク事象の発生シミュレーションを行ったり、統計的分析を行うことにより、
経営に与える影響を把握してリスクマネジメントに活用する。そうして平均的に発生すると予測される損失をEL(Expected Loss)、経営が許容しうる最大予想損失をVaR(バリュー・アット・リスク)、
VaRからELを引いたものをUL(Unexpected Loss)という。ELは期間利益で備え、ULは資本で備えることで企業経営は安定する。また、VaRは過去の一定期間の変動データに基づき、
将来のある一定期間のうちにある一定の確率の範囲内で被る可能性のある最大損失額を統計的手法で推定した値であり、客観性が高く説得力が高い。ただし、推定された値であるから使用に耐えられるかは統計的に検証する必要がある。
VaRは市場リスク、信用リスク、統合リスクの推定などに使われている。
\par 第三章ではVaRの計測手法についての解説が行われている。まずは市場VaRの計測についてである。市場VaRは特定の金融資産・負債をポートフォリオとして持っているとき、金利、株価、為替などの市場変動に伴ってどれぐらいの損失がどれくらいの確率で生じる可能性があるのか
を予測したいというニーズから生まれた。計測に当たり、金利、株価、為替などの市場変動をリスクファクターと呼び、リスクファクターの変動はモデルを簡潔にするために過去も現在も将来も互いに独立で同一の確率分布に従って起きるという想定をおいて分析を行うのが一般的である。
計測手法にはデルタ法がある。デルタ法ではリスクファクターは各地点において互いに独立な同一の正規分布に従って変動すると想定する。リスクファクターには一般的に金利、株価、為替の変化率を採用する。この方法を取ることで過去のデータから標準偏差を取るだけでよくなる。また、
リスクファクターの変動に関するポートフォリオ価値の変動額をデルタ、または感応度と呼び、これが一定であるという重要な仮定をしている。このとき、リスクファクターを横軸、ポートフォリオ価値を縦軸に取るとその2つの関係は直線で表される。しかし、前述の2つの仮定が成立しないと考えられるとき、
この方法は適用に難があると言える。また、デルタ法にはリスクファクターの観測データのとり方によってムービング・ウィンドウ法とルートT倍法の2つの手法がある。どれくらい先の将来まで予測するかの期間を保有期間というが、ムービング・ウィンドウ法では
保有期間T日のVaR計測においてリスクファクターのT日間変化率を観測データとして取得する。この方法ではーのリスクファクターのT日間変化率に関して、平均0、標準偏差が $\delta _r$の正規分布に従って変動すると仮定する。
また、リスクファクターの変化に対するポートフォリオ価値の変化額(デルタ)は$\Delta$で一定であると仮定する。
上記の仮定のもと具体的な計測式はT日間変化率の標準偏差を${\delta}_t$信頼係数を2.33として、99\%VaR = $\Delta \times$信頼係数 $\times {\delta}_t$となる。\par
次にルートT倍法について説明する。ルートT倍法は保有期間T日間のVaRを、リスクファクターの日時変化率を観測データとして計測する方法である。この方法を使う条件として、日時変化率は対数変化率で採用すること、日時変化率が互いに独立で同一の確率分布に従って変動することの2つを満たす必要があり、
満たされているときは日時変化率の標準偏差をルートT倍することでT日間変化率を求められる。これを保有期間調整、あるいはルートT倍ルールという。保有期間調整を使うと、VaR計測式は
信頼係数を2.33、T日間変化率の標準偏差を${\delta}_t$、日時変化率の標準偏差を${\delta}$として、99\%VaR = $\Delta \times 2.33 \times \sqrt{T} \times \delta$となる。
今まで紹介した2つの方法をまとめて分散共分散法というが、分散共分散法では観測データの変動が全地点で独立であることが前提となる。しかし、一般的にはT日間変化率には自己相関が観察されるため、ムービング・ウィンドウ法ではその前提が満たされない。ではルートT倍法のほうが優れているのかといえば一長一短である。
ルートT倍法では日時変化率お推移を煮ると、外見上の自己相関はほとんど気にならないほど小さいが、厳密な意味で独立であるとは確認できないことがわかっている。したがってルートT倍法の成立もあくまで近似的であり、保有期間が長くなるほど近似制度が落ちる。いずれの手法を使う場合でもバックテストをクリアすることが重要になる。
また、リスクファクターの変化では正規分布の仮定では必ずしも満たされない。正規分布における理論値よりも実際の相場変動の頻度のほうが大きいからである。また、金利水準によってデルタも変化するのでデルタ一定の仮定も満たされないことが多いため、分散共分散法で計測されたVaRはあくまで近似値として活用できるかを考えることになる。
リスクファクターが2つ以上のときも分散共分散法を適用することができる。その時の式はポーフォリオの現在価値の標準偏差を$\delta_{pv}$として99\%VaR = 信頼係数  $\times {\delta}_{pv}$となる。\par
分散共分散法では正規分布とデルタ一定を仮定していたがこの仮定を緩めると解析的なVaR算定が困難になる。しかしモンテカルロ・シミュレーション法を使うと、そういった場合でもVaRの算作成を定は可能である。モンテカルロ法ではリスクファクターの確率分布はどのように仮定してもよく、過去の観測データから確率分布の徳性を示す平均や分散などのパラメータを求めて確率分布の形状を特定する。
その後、コンピュータでその確率分布に従って乱数を発生させてリスクファクターの予測値をつくってその予測値に基づいてコンピュータにポートフォリオ価値を計算させる。
このようにしてヒストグラムを作って99\%点をとれば99\%VaRが求められる。ただしシミュレーションを行うごとに計測値が変動するため、収束に時間がかかる。
ヒストリカル法という計測法もある。この計測法ではリスクファクターの確率分布としては特定の分布を指定しないのが特徴である。その代わり、リスクファクターの過去の観測データを使ってポートフォリオ価値の変動を用いて現在のポートフォリオを過去の観測機関の間ずっと保有していたと想定して、リスクファクターの変動実績からポートフォリオ価値をさかのぼって計算した理論変動値をヒストグラム化して将来を予想する際の確率分布として代用する。このヒストグラムの99\%点を取るとそれが
99\%VaRである。欠点としてはデータ数が少なくなるため計測制度が低くなりやすいことがあげられる。
近年ではリスクファクターの変動が必ずしも正規分布に従っているとはいえないことからヒストリカル法が採用されることが増加している。\par
個別債務者がデフォルトすつ確率とその時の損失額がわかれば信用VaRが計量化できる。すべての債務者の信用状態に影響を与える共通要因(X)と特定の債務者の信用状態のみに影響を与える固有要因(Y)とがあるとする。
Xがひとつだけで標準正規分布に従って変動するとする。個別債務者の信用状態($Z_i$)にのみ影響を与える$Y_i$についても標準正規分布に従うとする。このモデルはXを一つとしていることから1ファクターモデルと呼ばれ、aをXのウエイト(感応度)として$Z_i = aX + \sqrt{1 - a^2}Y_i$と表せる。
$Z_i$が債務者のデフォルト確率に対応したある閾値以下の数値を取るときデフォルトし、損失が発生すると想定する。Xが複数個ある場合もあるが、その場合は各種パラメータの推定、検証は難しくなる。\par
業務全般に関して内部プロセス、人、システムが適切に機能せず事件事故等が発生したり、災害等の外生的事象が起きたときに被る損失にかかるリスクをオペレーション・リスクという。このリスクに関しては事件事故等が一定期間に起きる発生件数と一件あたりの損失金額の双方を確率変数とみなす。
このようにして得られた一定期間の損失額の分布を使ってオペリスクVaRを求める手法を損失分布手法と呼ぶ。\par
VaRは推定値であり、リスクマネジメントで活用するにはバックテストと呼ばれる統計的検証作業が必要になる。VaRの場合は計測したVaRを損失実績が超過した回数を数える。信頼水準から想定される超過回数を大幅に上回っていればモデルとしてふさわしくないことになる。そうした回数の基準としてスリー・ゾーン・アプローチというものがある。
250回中4回以下の超過ならモデルに問題はない、5回以上9回以下の超過ならモデルに問題がある可能性があるが決定的ではない、10回以上なら間違いなく問題があるとする方法である。この方法は2項分布の考え方に従っている。バックテスト実行後はストレス事象の洗い出しやモデル改善につなげることが重要である。VaRは
あくまで推定値であるから、損失を100\%的確に指標化することはできない。よって、その不完全性を保管するためにストレステストを行うのが一般的である。その方法としては過去のショック事象のデータをそのまま使用するなどといった、客観性重視の方法や将来のあり得る変動や損失等を自由に想定する柔軟性重視の方法がある。
リスクマネジメントの実務ではVaRの計測結果とストレステストの結果を比較してリスク量の上限を探り、リスクテイクの状況を評価することが重要である。また、先進的な企業や金融機関ではVaRでリスク枠を設定しストレステストの結果からリスク資本を配賦するという枠組みを構築することが多い。
VaRで対外的な説得力を増しつつ、余裕を見込んでリスク資本を配賦していることが主張できれば経営体力の安全確保を図っていることが明確に伝わり、大いに安心感を与えることができるからである。
内部監査はその企業の経営におけるリスクマネジメントプロセスが組織全体で有効に機能しているかをリスクベースで検証して経営陣に報告する。その意味では内部監査部署は組織ディフェンスにおいて牽制機能を発揮すると言える。リスクマネジメントプロセスを客観的に評価するには内部監査部署は各部署から独立している必要がある。
また、内部監査部署は監査結果から各部署に対してリスクマネジメントプロセスの見直しや改善を促すという役割を果たしている。リスクマネジメントが高度化しても内部監査が果たす役割・機能は変わらず、内部監査も高度化していく必要がある。内部監査でリスクマネジメントプロセスを検証するとき、形成者の視点で重要なリスクはなにかを捉え、基本方針や監査計画を策定することが重要である。
また、経営陣との重大リスクの認識共有とその重大リスクに人材を集中することがポイントになるので全体監査計画の策定前に経営陣との協議やリスク評価を行うのが重要である。個別監査の人員配分を決めるには予備調査を行って監査要点を絞り込む必要がある。また、重大なリスクのうち優先順位の高いものだけを監査するのではなく、一定期間のうちに
検証を終えるように計画し検証作業の網羅性を確保する必要がある。また、環境変化によって重要なリスクは変化しうるから、一度策定した計画も機動的に見直せるように補完措置が必要になる。
内部監査人にはリスク識別やコントロールの有効性評価、プロセスの改善を促すなどの能力、専門知識やヒューマンスキルが求められる。リスクが多様化・複雑化するにつれて一般的知識だけではいまや実効性のある内部監査ができないからである。しかし一人がすべての能力などを身につけられるわ絵でもないので、内部監査部門全体で
専門知識、スキルを身に着けた要因を量・質ともに十分確保する必要がある。\par 最後に、リスク計測にかかる検証ポイントについてである。金融庁の「金融検査マニュアル」にはリスク計測手法に関して内部監査で検証スべきポイントが簡潔に示されている。
ただし金融機関や企業によってリスクプロファイルやリスクマネジメントの方法やあり方が違うため、画一的に適用することが求められているわけではない。
\newpage
\par 内容としては、まずリスクの分類法から初めて聞いたものでこういった分類でリスクを分けるのか、また、リスクを計量化することがなぜ必要であるか、どのような利点があるのかなど勉強になった。統計の手法についても一般化された式を用いることやモデル作成は難しいものだと考えたが、今学んでいるポートフォリオ理論などの金融工学の理論や計量経済学の統計手法や考え方がこういったところで実務に役に立っていくのだな
と考えると少しだけ今している勉強の意味がわかったような気がした。また、金融庁から内部監査のためのマニュアルが公開されていることは高く評価できることである。公的機関によるマニュアルが公開されていることでチェック項目が明確かつ信頼性があり、ある程度統一されたものとなるため、高い技術を持った
内部監査人が調査することで監査の質にばらつきが出にくくなると考えられるからだ。この本には活用事例も掲載されていたため実際のVaR計測についてもイメージを深めることができた。
\par 計量化の理論と定式化が思ったよりもしっかりしていて、発展が始まってから歴史の浅い分野とは思えなかった。様々な手法が用意されており、それらを用いてその長所短所を補い合うことでより正確な推定ができるはずである。また、具体的なテストの基準なども存在しており
その点でもVaRというものは信頼性の高いものであると感じた。そのうえで、講義も含めて考えるとなぜ日本の企業がこうしたガバナンスの機構を一般的なものにしないのかという疑念がある。
問題が起こってからでは保守作業は遅い上に、予防できたはずの事象を発生させることで人件費や予防にかかる費用を明らかに超える金額的損失が生じる可能性があるにもかかわらずだ。そうやって大事件になったスルガ銀行などの前例があるにもかかわらず浸透しないのはかなり異常だし危険であると感じた。
そして、そうした内部監査やリスク管理システムを構築できるほどの体力がない企業が多いとしたらそれも危険であると考える。危険を抱えたまま進まなければならず、更にその危険を取り除く方法を取ることもできないからである。
もし企業にそうした機構を構築する体力がありながらも導入していないのであれば、経営に対しての透明性があまり高くないと感じる。大企業であればあるほどそういった面に対して株主の目が厳しいはずであるから、早急に導入するべきである。
世界で採用されている方法が結果を出せていないのであればその方法を採用しないことにも納得がいくが、結果が出ている以上、確かにその国によって環境や風土が違うという事実はありつつも挑戦してみることになんら不自然な部分はないだろうし、自分に株主となった経験がないため明確なことは言えないが株主にとっても企業経営の透明性が保たれ、投資のための情報が増えるなど
利点が増えるのだから反発も少なく導入できるのではないかと考える。特に現在はSNSなど、簡単に情報が拡散されるので企業が信用を失うのは一瞬である。そうした事態にならないためにも内部監査機構は設立されるべきであろう。
\par ところで現在の日本において専門知識を身につけた人材の社会からの評価は給与などを見てもあまり高くないのではないかと感じる。やりがいが報酬になり得るという価値観を否定するつもりはないが、業務に対して一番わかりやすい評価は金銭であるし、そうした手厚い評価を受けられる
ということをアピールしている企業で働きたいという人のほうが多いだろう。専門的な能力を持っている人にはそれ相応の評価がなされるべきである。
そうした高度な技術を持った人材が高い評価を受けられないとなれば、 高度教育を受けるインセンティブが減少することにもなりかねず、結果的にそのようなスキルを持った人間が減っていき、今以上の人材不足を引き起こし、内部監査を行う機構の構築を行おうとしたところで内部監査に必要な人間を確保できないという状態にもなりえず、
結果的に企業自身の首を絞めることにもなる可能性がある。そうならないように、日本の経済の構造として、専門的な知識を持った人により高い報酬を与えられる構造に変えていくことも必要であろう。
\end{document}