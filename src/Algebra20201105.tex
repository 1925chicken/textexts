%! Author = chicken1925
%! Date = 2020/11/05

% Preamble
\documentclass[11pt]{jsarticle}

% Packages
\usepackage{amsmath}

% Document
\begin{document}
2119116s 佐野海徳
\par HW27
\par 講義中に上がった例以外には、$\{\phi,\{1\},\{3\},\{1,3\}\}や\{\phi,\{2\},\{3\},\{2,3\}\},\{\phi,\{1\}\}, \{\phi,\{2\}\},\{\phi ,\{3\}\}$もイデアルである。
\par HW28
\par まず、kernelの定義から、kerとなる多項式はx = i,つまり$x^2 = -1$、言い換えれば$x^2 + 1$を因数に持つ。右辺の定義より右辺は$x^2 + 1$を因数に持ち、この条件を満たすとき左辺の条件を満たすので右辺は左辺を部分集合として含む。
    逆に、$x^2*1$の倍式、つまり$x^2+1$を因数に持つならば$x=i$を代入して0になる、つまりこの条件を満たすならばkernelの条件を満たす。故に左辺は右辺を部分集合として持つ。よって両辺はお互いを部分集合として含むかつお互いの部分集合なので
(左辺)=(右辺)が成り立つ。

\end{document}