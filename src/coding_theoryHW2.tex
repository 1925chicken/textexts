%! Author = chicken1925
%! Date = 2020/10/16

% Preamble
\documentclass[11pt]{article}

% Packages
\usepackage{amsmath}

% Document
\begin{document}
    佐野 海徳 20R13302 ICT.C209\\
\fbox{2.1}\\
$\boldmath{F}_2$では$1 + 1 = 0,1 + 0 = 0 + 1 = 0, 0 + 0 = 0$であり、$1 \times 1 = 1 , 0 \times 1 = 1 \times 0 = 0, 0 \times 0 = 0$である。
(a) 1 + 1 = 0\\
(b) 1 / 1 = 0\\
(c) 1(0110) = 0110\\
(d) 0(0110) = 0\\
(e) (011) + (001) = 010\\
(f) (111) / (111) = 1\\
\fbox{2.2}\\
    $\left(
    \begin{array}{cccccc}
    1 & 0 & 0 & 0 & \vdots & 1 \\
    1 & 0 & 1 & 1 & \vdots & 0 \\
    1 & 1 & 1 & 1 & \vdots & 1 \\
    0 & 0 & 0 & 1 & \vdots & 0
    \end{array}
    \right)$
$\to$
$\left(
\begin{array}{cccccc}
1 & 0 & 0 & 0 & \vdots & 1 \\
0 & 0 & 1 & 1 & \vdots & 1 \\
0 & 1 & 1 & 1 & \vdots & 0 \\
0 & 0 & 0 & 1 & \vdots & 1 \\
\end{array}
\right)$
$\to$
$\left(
\begin{array}{cccccc}
1 & 0 & 0 & 0 & \vdots & 1 \\
0 & 0 & 1 & 1 & \vdots & 1 \\
0 & 1 & 0 & 0 & \vdots & 1 \\
0 & 0 & 0 & 1 & \vdots & 1
\end{array}
\right)$
$\to$
$\left(
\begin{array}{cccccc}
1 & 0 & 0 & 0 & \vdots & 1 \\
0 & 1 & 0 & 0 & \vdots & 1 \\
0 & 0 & 1 & 0 & \vdots & 0 \\
0 & 0 & 0 & 1 & \vdots & 1
\end{array}
\right)$
となるから$x_1 = 1,x_2 = 1,x_3 = 0,x_4 = 1$である。\\
\fbox{2.3}\\
(a)基底は(01),(10)であり、次元は2、生成行列は$\left(\begin{array}{cc}
0 & 1 \\
1 & 0 \\
\end{array}\right)$ 最小距離は1、符号化率は$\frac{1}{2} \log_2 |3|$である。\\
(b)基底は(11111),(01001),(00100),(10010)であり、次元は4。生成行列は$\left (\begin{array}{ccccc}
1 & 1 & 1 & 1 & 1 \\
0 & 1 & 0 & 0 & 1 \\
0 & 0 & 1 & 0 & 0 \\
1 & 0 & 0 & 1 & 0 \\\end{array}\right)$であり、最小距離は1、符号化率は$\frac{1}{5}\log_2 |8| = \frac{3}{5}$である。\\
(c)(11011) + (11111) = (00100)だが(00100)は$C_3$上に存在しないので二元線系符号でない。\\
\fbox{2.4}\\
$\boldmath{F}_5$上の長さ5の繰り返し符号は{(11111),(00000)}であるが、0倍,1倍スカラー倍をかける操作をして両方の符号を生成できるのは(11111)だから求めるGは(11111)。\\
パリティ検査行列は定義より1が偶数個含まれる符号であるから\\
C = \{00000,11000,10100,10010,10001,11110,11101,\\11011,11011,10111,01111,01100,01010,01001,00110,00101,00011\}である。\\
\fbox{2.5}パリティ検査行列Hに適当な変形をほどこすと
$\left(
\begin{array}{cccccccc}
1 & 0 & 0 & 0 & 1 & 1 & 0 & 1 \\
0 & 1 & 0 & 0 & 0 & 1 & 1 & 1 \\
0 & 0 & 1 & 0 & 1 & 0 & 1 & 1 \\
0 & 0 & 0 & 1 & 1 & 1 & 1 & 0 \\
\end{array}
\right)$
となる。どの2列、3列を取り出しても異なるが、1列目、2列目、3列目、8列目を取り出すと線形従属であるから、(11100001)は符号語でありd(C)は4であることが示された。\\
\fbox{2.6}\\
(a)$Hy^t = (100)$であるから$s = (100)$である。\\
(b)以降はわかりませんでした。\\
\fbox{2.7}, \fbox{2.8}はわかりませんでした。\\
\fbox{2.9}\\
$G^{\prime} =
\left(
\begin{array}{ccccc}
1 & 0 & 0 & 0 & 1 \\
1 & 1 & 1 & 0 & 0 \\
0 & 0 & 1 & 1 & 0 \\
\end{array}
\right) \to
\left(
\begin{array}{ccccc}
1 & 0 & 0 & 0 & 1 \\
0 & 1 & 0 & 1 & 0 \\
0 & 0 & 1 & 1 & 0 \\
\end{array}
\right)
$\\となるからG = $\left(
\begin{array}{ccccc}
1 & 0 & 0 & 0 & 1 \\
0 & 1 & 0 & 1 & 0 \\
0 & 0 & 1 & 1 & 0 \\
\end{array}
\right)$。\\
(b) $H = \left(
\begin{array}{ccccc}
0 & 1 & 1 & 1 & 0 \\
1 & 0 & 0 & 0 & 1 \\
\end{array}
\right)$。\\
(c)Hは$C^{\perp}$の生成行列であるから$G^{\perp} = H$。\\
\fbox{2.10}パリティ検査行列の生成について復習していたとき、授業資料だけだとちょっと分かりづらかったです。コセットがあまりよく理解できませんでした。
\end{document}
