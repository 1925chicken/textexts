%! Author = chicken1925
%! Date = 2020/10/13

% Preamble
\documentclass[11pt]{jsarticle}

% Packages
\usepackage{amsmath}
\begin{document}
    2119116s 佐野 海徳\\
HW17 まず$\sigma$と$\tau$に同じ置換を選んだとき、それは恒等置換、つまりeになる。例えば$\sigma = (1,2),\tau = (1,3)$とすると、求める置換は$(1,2)(1,3) = (2,1)(1,3) = (2,3)$。
もう一つ例をあげる。$\sigma = (1,2,3),\tau = (2,3)$とすると、求める置換は$(1,2,3)(2,3) = (3,1,2)(2,3) = (3,1,3) = (3,1)(1,3) = (3,1)(3,1) = e$である。このようにして置換を求めると次の表のようになる。\\
\begin{table}[h]
\begin{tabular}{|l||l|l|l|l|l|l|}
      & e & (1,2) & (1,3) & (2,3) & (1,2,3) & (1,3,2) \\ \hline \hline
e      & e & (1,2) & (1,3) & (2,3) & (1,2,3) & (1,3,2) \\
(1,2) & (1,2) & (e) & (2,3) & (1,3) & (2,3) & (1,3) \\
(1,3) & (1,3) & (2,3) & e & (1,2) & (1,2) & (2,3) \\
(2,3) & (2,3) & (1,3) & (2,1) & e & e & (2,1) \\
(1,2,3) & (1,2,3) & (1,3) & e & (1,2) & e & e \\
(1,3,2) & (1,3,2) & (2,3) & e & (1,3) & e & e \\
\end{tabular}
\end{table}
\end{document}