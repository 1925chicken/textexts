%! Author = chicken1925
%! Date = 2020/12/02

% Preamble
\documentclass[11pt]{jsarticle}

% Packages
\usepackage{amsmath}
\usepackage{amsfonts}
% Document
\begin{document}
2119116s 佐野海徳\\
HW39\\
    ある複素数$a + bi (a,b \in \mathbb{R}, b \not 0)$を解に持つ方程式を考えると、共役な複素数$a - bi$も解である。ここで二次方程式の解と係数の関係より、この2数を解に持つ方程式は
$x^2 - (a + bi + a - bi)x + (a + bi)(a - bi) = 0$、つまり$x^2 - 2ax + a^2 * b^2 = 0$である。ここでa,bが実数であることから、$a^2, b^2, a^2 + b^2  \in \mathbb{R}$である。
    よって、すべての複素数は実数係数の2次方程式の解となるので、題意は示された。


\end{document}