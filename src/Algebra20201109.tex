%! Author = chicken1925
%! Date = 2020/11/09

% Preamble
\documentclass[11pt]{jsarticle}

% Packages
\usepackage{amsmath}

% Document
\begin{document}
2119116s 佐野 海徳\\
HW29
\par (n) $\supset$ (6)になるのは、6がnの倍数のとき、またそのときに限る。よってn = 1,2,3,6。 (1) = \{...0,1,2,3,4,5,...\}なので\{0,1,2,3,4,5\} = $Z / 6Z$はイデアル。
(2) = \{...0,2,4,...\}なので\{0,2,4\} $\subset$ Z/6Zはイデアル。 (3) = \{...0,3,...\}なので\{0,3\} $\subset$ Z/6Zはイデアル。 (6)からできるイデアルは{0} $\subset$Z/6Z。
よって答えは\{0\}, \{0,3\},\{0,2,4\},\{0,1,2,3,4,5\}の4つである。\\
HW30
\par 0次の元は,0と1の2つである。1次の元はx,x + 1の2つである。2次の元は$x^2,(x + 1)^2 = x^2 + 1,x(x + 1) = x^2 + x,x^2 + x + 1$の4つ。3次の項は$x \times x^2 = x^3,
x(x^2 + 1) = x^3 + x, x(x^2 + x) = x^3 + x^2,x(x^2 + x + 1) = x^3 + x^2 + x, (x + 1)x^2 = x^3 + x^2,(x + 1)(x^2 + 1) = x^3 + x^2 + x + 1,
(x + 1)(x^2 + x) = x^3 + 1$と$x^3 + x^2 + 1$の重複を除いて8個である。
よって、0次の元は0,1,1次の元はx,x + 1,二次の元は$x^2,x^2 + x + 1,x^2 + 1,x^2 + x$の4つ,3次の元は
$x^3,x^3+x^2,x^3+x,x^3+1,x^3+x^2+x,x^3+x^2+1,x^3+x^2+x+1,x^3 + x + 1$の8つである。
この内、既約元は0,1,x,x + 1,$x^2 + x + 1, x^3 + x^2 + 1,x^3 + x + 1$である。
\end{document}