%! Author = chicken1925
%! Date = 2020/11/17

% Preamble
\documentclass[11pt]{article}

% Packages
\usepackage{amsmath}

% Document
\begin{document}
2119116s 佐野 海徳\\
HW33
\par $/frac{1}{103} = 0.00970873786407766990291262135922330097......$となるので循環節の長さは34。
これは確かに$\varphi(103) = 103^1 - 103^0 = 102$の約数である。\\
HW34
\par まず1は明らかに原始根ではない。2についてmod17で計算すると、2,4,8,16,15,13,9,1で位数が16でないので原始根でない。3について同様に計算すると、\\
3,9,10,13,5,15,11,16,14,8,7,4,12,2,6,1で位数は16。よって原始根である。ほかの元についても同様にすると、位数16となる元は3,5,6,7,10,11,12,14の8つ。
よってmod17の原始根はこの3,5,6,7,10,11,12,14の8つである。



\end{document}