%! Author = chicken1925
%! Date = 2020/12/29

% Preamble
\documentclass[11pt]{jsarticle}

% Packages
\usepackage{amsmath}
\usepackage{url}
% Document
\begin{document}
    2119116s 佐野海徳
    \par 第四北越フィナンシャルグループのディスクロージャー誌を読みこの銀行の特徴などをまとめる。
    \par このグループは第四銀行と北越銀行という新潟の地方銀行2行の合併によって生まれたものであり、2021年1月から第四北越銀行として統合される。この合併によって新潟県内の預貯金シェアは4割、貸出金シェアは5割以上を達成している。
    また、JCRによる格付けではA+となっている。地方銀行10行による地域の持続的成長や金融システムの高度化、参加グループの企業価値増大などに寄与する施策の立案・推進を目的とした地銀広域連携である「TSUBASAアライアンス」に参加しており、広域かつ大規模な
    連携によるスケールメリットを活かした事業を展開することを可能にしている。また、連結子会社一覧を見ると子会社における関節所有を含めた議決権比率は6つのうち4つで66\%を超える割合となっており、上場はしているものの殆どを自己株式で持っていると言っていいだろう。
    また、FP1級・2級を取得している行員・社員も増加しており、プロとしての専門祭を高め、より信頼して取引を行える体制づくりに励んでいる。
    このグループの経営理念は「信頼される金融グループとしてみなさまの期待に応えるサービスを提供し地域社会の発展に貢献し続けること」、「変化に果敢に挑戦し新たな価値を創造する」というものである。
    現在、その実現に向け「金融仲介機能及び情報仲介機能の向上」、「経営の効率化」、「グループ管理体制の高度化」に取り組んでいる。
    経営指標を見ると消費増税や大型台風の影響、新型コロナウイルス感染症の世界的拡大の影響を受け連結当期純利益は大きく下がっており、効率性指標も下がっているが、成長性指標、健全性指標、
    また、ポートフォリオ変革に向けた指標の殆どが上昇を見せていることから一概に経営が悪くなっているとは言えない。
    さらに前述のポートフォリオ指標の改善に加え成長性指標の中の中小企業向け貸出平残の増加が見られ、取り組みの成果があらわれていると言える。
    また、自己資本比率も約9\%と国内基準を大きく超えていて経営に対して不安定性はあまり大きくないように感じる。ただし、不良債権割合が2019年3月期から2020年3月期に1.28\%から1.42\%に推移しており、一時的であればよいがやはり心配な面も見える。
    また、地域貢献指標も目標として設定した値を超えているものが多く、地域の人々の期待に応えるという面では良好な経過をたどっていると言える。具体的な地域貢献としては自治体等への金銭寄付等や、地域の学校へ図書やスポーツ用品などの学校用備品を寄贈するなどしている。
    また新潟県出身の大学生に対して給付奨学金事業を行っている。
    地域や利用者に対する積極的な情報発信も行っている。
    ホームページによる情報発信やセミナーによる情報発信を通して、情報公開し利用者に対して利益をもたらそうとする姿勢と企業としての情報公開義務をともに果たしていると言え、特筆すべきこととして
    2020年3月期にはセミナー開催階数が格段に増えており、目標達成に向けての戦略強化を行っていると推測できる。

    \par 金融仲介機能及び情報仲介機能の向上に関しては金融業だけでなく他領域への業務拡大を行うのが主な手段となっている。
    1つ目は地域商社である。アンテナショップのオープンや観光振興のために、訪日外国人向けの多言語翻訳ツール等の導入の提案を行うなどの事業を行っている。
    講義で出た、たましんの例であったような人材コンサルティングにも業務を拡大しており、やはりグループとしての手応えはコンサルティング事業において大きいものであるのだろうかと推測している。
    また、自分の住む地域で親しみのある企業がそういった事業を行うことで信頼感や信用が大きくなり、新たに他業種参入するハードルが低いことも こうした業種に参戦する理由の一つになるのだと考える。
    この2つの事業の共通点として、地域との結びつきの強化、という目的が見えることがある。あくまでも第四北越フィナンシャルグループは新潟県の地方銀行によるグループであり、利用者の多くが新潟県民である、
    そのことから、新潟県の地域振興を行うことが回り回って自分たちの利益につながるという図式も見える。実際、県内での少子高齢化はとどまることを知らず進行しているため、より新潟県の観光を振興するなどして県内への人の流入を目指し
    ビジネスチャンスを探し、また、地域活性化を行うことで融資等の活発化を行うことが重大目標とも言える。
    \par 役員構成を確認すると、代表取締役含めて社内の取締役は9人、社外取締役は4人となっており、この14人の中で監査等委員は5人(社内1人、社外4人)という構成になっている。
    社内取締役と社外取締役の人数比率をみると残念ながらガバナンスの国際基準に達しているとは考えられない。コーポレートガバナンス報告書によれば
    「当社の取締役会は、経営の意思決定・監査機能を適切に果たすため、公認会計士、弁護士、会社経営者、大学教授の4名の独立社外取締役を含む、知識、経験、能力をバランスよく備えた多様性のある構成」という記述があるが、少し物足りない。
    具体的なガバナンス体制を見ていくと、監査等委員会が取締役の監査・監督を行っているが、
    監査部が取締役の指揮・命令で動き、監査等委員会と監査部が協力・連携するという 構成となっている。監査部の独立性が保証されていると言えないのは残念である。
    ただし、体制図によるとリスク委員会やコンプライアンス委員会、会計監査人は取締役会等から独立しているという見方ができ、そこは評価されるべき点であると考える。
    また、金融庁による責任ある機関投資の諸原則(日本版スチュワードシップ・コード)の受け入れ表明をしたということで適正に運用されていれば一定のコーポレートガバナンスのレベルが保たれていると予想される。
    信用リスク管理体制の記述を読むと、「信用リスクの定量化を行い、自己資本に見合った信用リスクリミットの設定等を行っています」とあることからVaRを活用しているというふうに読み取れる。
    \par 事業性評価に向けた取り組みについて見ていく。事業性評価に基づく融資を行う与信先数、融資残高は2020年3月期にはそれぞれ約全体の14\%,28\%となっており、単に融資を行っているわけではない。また、ローカルベンチマークを示して対話を行うこと、
    外部専門家を活用する「技術協力室」による取引先訪問、技術的相談の対応などに応じるというように地元の産業支援を手厚く行っていることが読み取れる。また、取引先の本業を支援するため研修を実施し資格を取得するなど、
    取引先にただ融資するのではなく取引先に寄り添い、地域住民からの信頼を得られるような業務活動を行っている。
    \par 以上のことから、第四北越ファイナンシャルグループは特に危ないと言う印象を受けず、かと言って急成長をしているわけでもないという状態である。また、主に地域密着型経営を推進していると言える。その上で単なる融資に終止するのではなく、創業から技術的相談、取引先訪問、情報提供や人材マッチングなど顧客サポートを手厚く行っており、顧客との信頼関係を築く体制になっている。
    また、観光振興を行うことで新潟の経済を活性化し人材流入を目指す、学校への寄付などで潜在的顧客である地域住民からの信頼を得るということも可能にしており、地域住民の期待に応えるという面では大きく評価できると考える。ガバナンス体制は少し独立性が物足りない。いくら地方銀行といえども世界基準のガバナンス基準に合わせることは決して無駄ではなく、むしろメリットにもなるはずでありそうした方向の
    努力も行ってほしいと一利用者として考える。\\
参考:\url{https://www.dhfg.co.jp/esg/governance/pdf/governance_report_200624.pdf}\\
    \url{https://www.dhfg.co.jp/esg/governance/pdf/governance_200131.pdf}\\
    \url{https://www.dhfg.co.jp/financial/ir/disclo/pdf/dicro202003_siryohenn.pdf}\\
    \url{https://www.dhfg.co.jp/financial/ir/disclo/pdf/tougouhoukoku_dicro202003_honnpenn.pdf}
\end{document}