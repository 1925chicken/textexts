%! Author = chicken1925
%! Date = 2020/11/05

% Preamble
\documentclass[11pt]{jsarticle}

% Packages
\usepackage{amsmath}
\usepackage{cases}
\usepackage{amsfonts}
\newcommand{\argmin}{\mathop{\rm arg~min}\limits}
% Document
\begin{document}
佐野 海徳 20R13302 ICT.C209\\
\fbox{K.1}
\par(a)符号長は12、最小距離は100001011001と001101010010の組、あるいは111010001100と110110110101の組における6である。符号化率は$\frac{1}{12}\log_2|4| = \frac{1}{6}$である。
\par(b)受信語$\vec{r}$とのハミング距離が最小になる符号語$\vec{c}$を複合結果とする複号法である最小距離復号法における復号結果のこと。正確には$\vec{c}^{(MD)}\vec{r} = \argmin_{\vec{c} \in C} d(\vec{c},\vec{r})$で定義される。
\par(c)$\vec{c_t}^{(BD)}(\vec{r}) = \vec{c}\in C (d(\vec{c},\vec{r})\leq t$ となる符号語が唯一存在する)
$\vec{c_t}^{(BD)}(\vec{r}) = $error (そんな符号語は存在しないため復号失敗) が定義である。
\par(d)dを偶奇で場合分けすると、$2t < d$は$
\lfloor d_c \rfloor
 =
\left\{
\begin{array}{l}
d_o (d = 2d_o + 1) \\
d_e(d = 2d_e + 2)
\end{array}
\right.$
,$\frac{d}{2} = \left\{
\begin{array}{l}
d_o + \frac{1}{2} \\
d_e + 1 (d = 2d_e + 2)
\end{array}
\right.$。
ここから$t < \frac{d}{2}$は$
\lfloor \frac{d - 1}{2}\rfloor
$と同値。つまり、各符号語から$\frac{d}{2}$より小さい半径にある受信語の集合に交わり
がないので、$\frac{d}{2}$より少ない数の誤りは訂正できると言える。ここで逆に異なる2つの符号語$\vec{c_1},\vec{c_2}$の$\frac{d}{2}$より小さい半径に共通して
含まれる受信語$\vec{r}$が存在したと仮定するとdを符号Cの最小距離として
$d \geq d(\vec{c_1},\vec{c_2}) \\
    \geq d(\vec{c_1},\vec{r}) + f(\vec{r},\vec{c_2}) \\
    \geq \frac{d}{2} + \frac{d}{2} = d$でd = dとなり矛盾が起こる。\\
次に半径tの限界距離符号は$t \leq \lfloor \frac{d - 1}{2} \rfloor + 1$に対して$\lfloor\frac{d - 1}{2}\rfloor$個より多い誤りを訂正できないことがあると示す。
最小距離をdを与える符号語のペアを$\vec{c_1},vec{c_2}$とする。符号語$\vec{c_1}$を送信し、$\vec{c_1}$の中で$\vec{c_2}$と異なるd個の要素のうちt個が$\vec{c_2}$の要素に変わった受信語$\vec{r}$を受信したとする。
このとき$d(\vec{r},\vec{c_1}) = t, d(\vec{r},\vec{c_2}) = d - t$である。$\lfloor \frac{d - 1}{2} \leq \frac{d - 2}{2}$だから
$d(\vec{r},\vec{c_1}) = t \leq \lfloor\frac{d - 1}{2}\rfloor + 1 \leq \frac{d - 2}{2} + 1 \leq \frac{d}{2}$
つまり $t \leq \frac{d}{2}$となる。これより$t = d(\vec{r},\vec{c_1})\leq d - t = d(\vec{r},\vec{c_2})$をえる。これは$\vec{r}から半径t以内に少なくとも2つの符号語\vec{c_1},\vec{c_2}$が存在することを意味するので、
復号エラーとなる。つまり、誤り訂正能力から$t \geq \lfloor \frac{d - 1}{w} \rfloor$ならば半径tの限界距離復号は任意の符号語$c \in C$を送信した場合にt個の任意の誤りを訂正できるから
$2t < d$となるtに対して$d(\vec{c},\vec{r})$となる符号語$\vec{c}$は存在するとしたら一意である。
\newpage
佐野 海徳 20R13302 ICT.C209\\
\fbox{K.2}
\par(a)Cの各符号語からハミング距離t以下のベクトル全体は互いに交わりがない。実際、交わりがあると仮定すると、最小距離がdであることに矛盾。したがってユニオン限界を投資機で満たし、$\# \cup\{\vec{c} \in C\}B(\vec{c},t) = \sigma_{\vec{c},C}\#B(\vec{c},t) = |C|V_2(n,t)$が成り立つ。
よって左辺の集合は$\mathbb{F}^n_{2}$に含まれるまたは等しいので$2^n \leq |C|V_2(n,t)$となる。ので$M \geq \frac{2^n}{V_2(n,t)},t = \lfloor\frac{d - 1}{2}\rfloor$が成り立つ。よってこれは必要条件を与える。
\par(b)VG限界を言い換えると、$M - 1 \textless \frac{2^n}{V_2(n,d - 1)}$が成り立つ(M - 1, n,d)符号に対して、(M - 1,n,d)符号Cは最大ではない。したがって、Cに最小距離をdに保ったまま符号語を一つ増やせる、すなわち(n,M,d)符号が存在する。よって、これは十分条件である。
\par(c.1)$B \leq A \frac{4}{3}\pi r^3$
\par(c.2)容器にA個の球雨が入っているとすると、このとき各球から距離r以下の空間には互いに交わりがないのでユニオン限界を等式で満たし、すべての球が占有する空間の体積は各球の体積の和に等しい。更にすべての球が専有する空間の体積は容器の体積を超えないので
$B \leq A \frac{4}{3}\pi r^3$となる。最大充填でも上の議論は成り立つので$B \leq A(B,r)\frac{4}{3}\pi r^3$となる。
\newpage
佐野 海徳 20R13302 ICT.C209\\
\fbox{K.3}
\par(a)$C_1$は零ベクトルが含まれていないので線形符号でない。$C_2$は二元線形符号であり、次元は3、生成行列は$\left(
\begin{array}{ccccc}
1 & 0 & 1 & 1 & 0 \\
0 & 1 & 1 & 0 & 1 \\
1 & 1 & 1 & 1 & 1
\end{array}
\right)$である。最小距離は1、符号化率は$\frac{3}{5}$である。
\par(b) 左辺を計算すると$\left(
\begin{array}{c}
    x_1 \\
    x_1 + x_3 + x_4 \\
    x_1 + x_2 + x_3 + x_4 \\
    x_4
\end{array}
\right)$である。右辺の値から、$x_1 = 1,x_4 = 0$は自明。これを用いると$x_2 = 0,x_3 = 1$も同様にしてわかる。
\par(c)C = \{000,111\}である。000に何をかけても0であるから、111と内積を取って0になるベクトルを考えると、長さ3で1を偶数個含むものが当てはまる。よって、$C^{\perp} = \{000,011,101,110\}$である。
\newpage
佐野 海徳 20R13302 ICT.C209\\
\fbox{K.4}
\par(a)$4 \times 4$の対角行列の右にHの1~4行目を転置したものを書いたものがGであるから$
\left(
\begin{array}{ccccccc}
1 & 0 & 0 & 0 & 1 & 1 & 0 \\
1 & 1 & 0 & 0 & 0 & 1 & 0 \\
0 & 0 & 1 & 0 & 1 & 0 & 1 \\
0 & 0 & 0 & 1 & 1 & 1 & 1
\end{array}
\right)$である。
\par(b)uGを計算すると、求める符号語は(1010011)である。
\par(c)受信語をrとする。$Hr^t$を計算すると、(010)である。これはHの第六列に等しい。よって、推定誤りベクトルを$\hat{e} = (0000010)$として、推定送信語を求めると、$\hat{c} = r - \hat{e} = (1111111)$である。
よって求める推定符号語は(1111111)。
\par(d)符号長は7。次元は4。符号化率は$\frac{k}{n} = \frac{4}{7}$。
\par(e)Hのどの二列を取り出して計算しても$(000)^t$とは異なるため、最小距離は3以上である。
\par(f)1列目、2列目、3列目を取り出して計算すると、$(000)^t$となる。よって、これらは線形従属であり最小距離は3以下である。
\par(g)問題のハミング符号Cは$(n - 2^3-1,M = 2^{2^m - 1 - m},d = 3)$符号である。さらにd(C)= 3より、t(C) = 1,$V_2(n,t) = n + 1$となる。これらはハミング限界を投資機で満たし、Cは完全である。よって題意は証明された。\newpage
佐野海徳 20R13302 ICT.C209\\
\fbox{K.5}
\par(a)1つめは加法単位元を0とし、逆元は任意の元xに対して-xを取ることでx-x = 0を満たすので,-x,結合性については任意の元a,b,c$\in \mathbb{Q}$に対し(a + b) + c = a + b + c = a + (b + c)が成り立つので成立。
よって単位元0と逆元-xが存在して、結合性も満たすので群となる。2つめに対して、結合則を考えると、任意の元$a,b,c \in \mathbb{R}$に対して$(a \times b) \times c = ab \times c = abc, a \times (b \times c) = a \times bc = abc$が成立するので成立。
よってこれは群であり、乗法単位元を1,乗法逆元を$\frac{1}{x}$とする群である。3つめに関しては、正則でない行列には逆行列が存在しないので逆元が存在しない。よって、これは群にならない。4つめは群となる。$f,g \in S(x)$に対してXの異なる2元x,yを取ればfは単射だから
$f(x) \neq f(y)$、またgも単射だから$g(f(x)) \neq g(f(y))$、つまり$(g \dot f)(x) \neq (f \dot g)(y)$となり合成写像$g \dot f$は単射。次に$z \in X$をとるとgは全射。よってg(y) = zを満たす$y \in X$が存在する。更にfも全射であるからf(x) = yとなる$x \in X$
が存在する。よって$z = g(y) = g(f(x)) = (g \dot f)(x)$が成り立ち合成写像$g \dot f$は全射である。したがって$g \dot f \in S(X)$。$f,g,h \in S(X)$に対して$(h \dot (g\dots f))(x) = h((g \dots f)(x)) = h(g(f(x)))$、$((h \dot g) \dot f)(x) = (h \dot g)(f(x)) = h(g(f(x))) (x \in X)$
よって結合法則が成立する。X上の恒等写像$1_X:x \mapsto x(x \in X)$はS(X)の元であって$(1_X \dot f)(x) = 1_X(f(x)) = f(x) = f(1_X(x)) = (f \dot 1_X)(x) (x \in X)$すなわち$1_X \dot f = f \dot 1_X = f$だから$1_X$はS(X)は単位元。
S(X)の元fは全単射であるから、逆写像$f^{-1}$が存在して$f^{-1}$が存在して、$f^{-1}$も全単射。よって$f^{-1} \in S(X)$。
$(f^{-1}\dot f)(x) = f^{-1}(f(x)) = x = f(f^{-1}(x)) = (f \dot f^{-1})(x)(x \in X)$。すなわち$f^{-1} \dot f = f \dot f^{-1} = 1_X$だから$f^{-1}$は逆元である。
\par(b.1)$e,e^{\prime}$を単位元とする。$e,e^{\prime}$が単位元であることから$ee^{\prime} = e^{\prime}, ee^{\prime} = e$となる。よって、$e^{\prime} = e$であり、単位元は一意に存在する。
\par(b.2)$a^{\prime} ,a^{\prime\prime}$をaの逆元とする。このとき、逆元の存在から、$a^{\prime} a = e$となる。両辺に右から$a^{\prime\prime}$をかけると、(左辺)$a^{\prime\prime} = (a^{\prime} a)a^{\prime\prime} = a^{\prime}(a a^{\prime\prime}) = a^{\prime} e = a^{\prime}$、
(右辺)$a^{\prime\prime}$ = $ea^{\prime\prime} = a^{\prime\prime}$となり、結局$a^{\prime} = a^{\prime\prime}$だから、逆元は一意に存在する。
\par(c)$99221 = 97343 \times 1 + 1878, 97343 = 1878 \times 51 + 1565, 1878 = 1565 \times 1 +313, 1565 = 313 \times 5 + 0$だから、求める最大公約数は313。\newpage
佐野海徳 20R13302 ICT.C209 \\
\fbox{K.6}
\par (a) 1つ目の式は1 + 1 = 0, 2つ目の式は$(1 + x^2 + x^3) + (1 + x + x^2) = x + x^3$,3つめの式は$(1 - x)(1 + x) = 1 - x^2$,4つめの式に対して
$(1 + x)(ax^5 + bx^4 + cx^3 + dx^2 + ex + f) + g = x^6 + x^2 + 1$という式を考える(a,b,c,d,e,f,gは0か1のどちらか)。(左辺) = $ax^6 + (a + b)x^5 + (b + c)x^4 + (c + d)x^3 + (d + e)x^2 + (e + f)x + (f + g)$であり、右辺と係数を比較すると、
a = 1,b = 1,c = 1,d = 1,e = 0,f = 0,g = 1となる。よって、商は$x^5 + x^4 + x^3 + x^2$,あまりは1である。
\par (b.1)
\par A. $[010]$ $\times$ $[010] = (x^2 \times x^2) mod p(X) = x^3 + x$。
\par B. $[010]$と乗算をして$[100]$となるものなので、計算すると、$[011]$が答えである。
\par C.$[011]\times[010]^{-1} = [011] \times[011] = [110]$である。
\par (b.2)
計算すると、$[100] \alpha_1 + [011] \alpha_2 = [010], [100]\alpha_1 + [101]\alpha_2 = [101]$。これをもとに演算表から条件を満たすものを探すと、$\alpha_1 = [100],\alpha_2 = [011]$。
\par (c)pは素数なので$gcd(a,p) = 1$。よって、$y,z \in \mathbb{Z}$が存在して$ ay + pz = 1 $となる。これより、[a][y] = [1]となって、この$[y] \in \mathbb{Z} / <p>$が[a]の逆元である。\newpage
佐野 海徳 20R13302 ICT.C209\\
\fbox{K.7}
\par (a).1 符号長は5、次元は3、$\mathbb{F}$上の線形符号では符号化率は$\frac{k}{n}$が成り立つので$\frac{3}{5}$。最小距離は$n - k + 1 = 5 - 3 + 1 = 3$。
\par (a).2 定義より各項を計算し、mod7で剰余をとると、G = $\left(
\begin{array}{ccccc}
ここ1 & 1 & 1 & 1 & 1 \\
0 & 1 & 2 & 3 & 4 \\
0 & 1 & 4 & 2 & 2
\end{array}
\right)$
。
\par(a).3 (403)G = (40233)。
\par (b)$f(X)$から$(f(\alpha_1),......f(\alpha_n))$に移す写像$\phi$は線形全単射であったから、Cの非ゼロ符号語は
$\vec{x}(f) = (f(\alpha_1),......f(\alpha_n))) \neq (0,...,0)$ with $f(X)(\neq 0) \in \mathbb{F}_q[X;k]$と書ける。
この非ゼロ符号語の重み、言い換えると$f(\alpha_i) \neq 0$となる$i(1 \geq i \geq n)$の個数を調べる。
零多項式は零符号語となる。つまり、$\phi:\mathbb{F}_q[X;k] \in 0 \mapsto (0,...0) \in \mathbb[F]{^n}{_q}$であるから対偶を考えると非零符号語は非零多項式から
生成されることがわかる。また、次数k未満の非零多項式$f(X)$の根の個数はk未満であることと、$\alpha_1,...,\alpha_n$が全て異なることから、$f(\alpha_i) = 0$となる
iの個数はk未満であることがわかる。言い換えると$f(\alpha_i)$の個数はk未満であることが分かる。言い換えるとm$f(\alpha_i) \neq 0$となるiの個数はn - k + 1以上である。
こうして$\omega(\vec{x})\leq n - k + 1$が示された。よってこのことから題意が証明された。\newpage
佐野 海徳 20R13302 ICT.C209 \\
\fbox{K.8}
\par(a)1つ目は巡回符号である。次元は1,符号長は4。
2つ目は巡回符号である。次元は4,符号長は4。
\par(b)$(1010)$によって生成される巡回符号を列挙すると、$(1010),(0101)$の2つである。
\par(c)$x^4 - 1$を素因数分解すると、$x^4 - 1 = (x+ 1)(x - 1)(x^2 + 1) = (x + 1)(x + 1)(x + 1)^2 = (x + 1)^4$である。ここで自明なものを除くと$g(X) = 1 + X,(1 + x)^2,(1 + x)^3$の3つである。
\par(e)まずCが線形符号であること、つまり、$c(X),d(X) \in C$に対して$u(X),v(X) \in \mathbb{F}[X;k]$が存在して$c(X) = u(X)g(X),d(X) = v(X)g(X)$と書ける。$a,b \in mathbb{F}$に対して$ac(X) + bd(X) = au(X)g(X) + bv(X)g(X) = (au(X) + bv(X))g(X) = au(X) + bv(X) \in \mathbb{F}[X;k]$となるので
$ac(X) + bd(X) \in C$となる。
次にCが巡回性を満たすと示す。右巡回シフトで閉じていることを示せば左巡回シフトでも同じようにできるので右巡回シフトで閉じていることを示せば十分である。言い換えれば$c(X) \in C$に対して$Xc(X) mod X^n - 1 \in C$である。
mod$X^n- 1$ で$Xc(X) \equiv Xc(X) - c_{n - 1}(X^n - 1) \equiv Xu(X)g(X) - c_{n - 1}h(X)g(X) \equiv (Xu(X) - c_{n - 1}h(X))g(X)$が成り立つ。ここで多項式f(X)のk次の係数を$coef(f(X);k)$とかくと$coef(Xu(X);k) = coef(c_{n - 1}h(X)) = c_{n - 1}$だから、
$coef(Xu(X);k) = coef(c_{n - 1}h(X)) = c_{n - 1}$なので$Xu(X) - c_{n - 1}h(X) \in \mathbb{F}[X;k]$となり、主張が導かれた。\newpage
佐野 海徳 20R13302 ICT.C209\\
\fbox{K.9}
\par(a)$\alpha^0 = (100),\alpha^1 = (010), \alpha^2 = (001), \alpha^3 = (110), \alpha^4 = (011), \alpha^5 = (111), \alpha^6 = (101)$となる。
ここで、演算表より、$\alpha^{50} = \alpha^{6 \times 8} \alpha^2  =\alpha^{5 \ times 4} \alpha^2 = \alpha^{3\times 2} \alpha^2 = \alpha^1$、$\alpha^{100} = \alpha^{6 \times 16} \alpha^4 = \alpha{5 \times 8} \alpha^4 = \alpha^{3 \times 4} \alpha^4
 = \alpha^{6 \times 2} \alpha^4 = \alpha^{5} \alpha^4 = \alpha^2$となり、べき表現は$\alpha^1 + \alpha^2$,ベクトル表現は$(010) + (001) = (011) = \alpha^4$。
\newpage
佐野 海徳 20R13302 ICT.C209\\
\fbox{K.10}
\par (a) (1)は$a^2$,(2)は $ a^{2\times 2} = a^4$,(3)は $a^{4 \times 2} = a^8$,(4)は$a^{3 \times 2} = a^6$,(5)は$a^{6 \times 2} = a^(12)$,(6)は$a^{12 \times 2} = a^9$,
(7)は$a^{5 \times 2} = a^10$,(8)は$a^{7 \times 2} = 14$,(9)は$a^{14 \times 2} = a^13$,(10)は$a^{13\times 2} = a^11$,(11)は$1 + X + X^4$
,(12)は$1 + X + X^2$,(13)は$1 + X^3 + X^4$。
\par (b)求める生成多項式は$a,a^2,a^3,a^4$を根に含む次数最小のモニック多項式だからg(X) = $(1 + X^2 + X^4)(1 + X + X^2 + X^3 + X^4)$となる。
\par (c)求める生成多項式は$a,a^2,a^3...a^6$を根に含む次数最小のモニック多項式であるからg(X) = $(1 + X + X^2 + X^3 + X^4)(1 + X + X^4)(1 + X + X^2)$。\newpage
佐野 海徳 20R13302 ICT.C209 \\
\fbox{K.11}
\par あまりに大量のデータを送受信することになると、コンピュータやサーバのメモリリミットに引っかかることがあると思うのですが、メールなどで打てる語数が決まっているというのは頻繁に使用されるサービスであり、多くの人が使用するものであるから
1つ1つのデータを安全にかつすばやく送るためのトレードオフとして仕方のないことなのでしょうか。modで符号を管理するのはコンピュータ上で扱える数値の最大が決まっているからなのかと考えました。
また、HashTableを使うことでも復号というものは実施できるのではないかと考えていますが、こうした理論が存在するということはなにかこちらの方法のほうがHashtableが大きくなることによるメモリ使用量の増大など、利点があるからなのだろうかとも思っています。
\end{document}