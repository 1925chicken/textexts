%! Author = chicken1925
%! Date = 2020/10/29

% Preamble
\documentclass[11pt]{jsarticle}

% Packages
\usepackage{amsmath}

% Document
\begin{document}
    \par(1).b)
    \par(1).c)現実では株価指数への組入れの際にインデックスファンドが大量購入し組み入れられる銘柄の株価は上昇するが、効率的市場仮説が成り立っているとき株価指数への組入れなどは株価のファンダメンタルズに影響を与えないはずなので株価の需要曲線は水平になる。よって需要曲線は右下がりにはならない。
    \par(2).a)資産3は資産1を2単位、資産2を1単位購入すると全く同じペイオフを複製できる。このとき、組み合わせて作られた資産の価格は$2 \times 0.6 + 1.4 = 2.6$であるから、どちらを買っても同じ金額であり、同一ペイオフを作り出す資産の金額が等しいので無裁定条件を満たしている。
    \par(2).b)複製可能な資産3を除いた資産は2つで、この経済の状態は2つであるから、この2資産からなるペイオフ・ベクトルが一次独立であればよい。
    $
    \left(
    \begin{array}{cc}
        0.5 & 1 \\
        1   & 3
    \end{array}
    \right)
    = \left(\begin{array}{cc}
                0.5 & 1 \\
                0   & 1
    \end{array}
    \right)\\
    = \left(
    \begin{array}{cc}
        0.5 & 0 \\
        0   & 1
    \end{array}
    \right)
    = \left(
    \begin{array}{cc}
        1 & 0 \\
        0 & 1
    \end{array}
    \right)
    $\\
    となり、rankが2であることから一次独立性が示され、この市場は完備であるといえる。
    \par(3).a) 一期間での期待値は$\frac{15 - 5}{2} = 5$であり、分散は$\frac{1}{2}(15 - 5)^2 + \frac{1}{2}(-5 - 5)^2 = 100$である。
二期間での期待値は$\frac{1}{4}(30 + 10 + 10 - 10) = 10$であり、分散は$\frac{1}{4}\{(30 - 10)^2 + (10 - 10)^2 + (10 - 10)^2 + (-10 - 10)^2\} =200$である。
\par(3).b)求める比率は$\frac{5}{100}\frac{1}{RRA} = \frac{1}{20 \times RRA}$である。
\par(3).c)求める比率は$\frac{10}{200}\frac{1}{RRA} = \frac{1}{20 \times RRA}$である。
\par(3).d)一期間での利益変動の確率には変化がないので一期間の期待値、分散は変化がない。二期間での変動を考えると、期待値は$\frac{1}{8}\{30 + 3 * 10 + 3 * 10 - 10\} = 10,分散は\frac{1}{8}(30 - 10)^2 + \frac{3}{8}(30 - 10)^2
+ \frac{3}{8}(10 - 10)^2 + \frac{1}{8}(-10 - 10)^2 = \frac{1}{8}(400 + 400 * 3 + 400) = 250$である。また、b)の解答に変化は起こらない。c)の解答は$\frac{10}{250}\frac{1}{RRA} = \frac{1}{25 \times RRA}$となる。
\par
\end{document}