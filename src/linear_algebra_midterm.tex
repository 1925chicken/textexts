%! Author = chicken1925
%! Date = 2020/11/03

% Preamble
\documentclass[11pt]{jsarticle}

% Packages
\usepackage{amsmath}

% Document
\begin{document}
2119116s 佐野海徳\\
\par1
\par(1)\\
(与式) = $\left(
\begin{array}{ccc}
    1 & 9 & 2 \\
    -2 & 1 & 3
\end{array}
\right)$ +
$\left(
\begin{array}{ccc}
    0 & -3 & 6 \\
    3 & 3 & -12
\end{array}
\right)$ =
$\left(
\begin{array}{ccc}
    1 & 6 & 8 \\
    1 & 4 & -9
\end{array}
\right)$\\
(2)\\
$1 \times 3$行列と$3 \times 3$では型が異なるので引き算は計算不可能である。\\
(3)\\
$1 \times 4$行列と$3 \times 3$行列では左の行列の列数と、右の行列の行数が異なるので掛け算は定義されておらず計算不可能\\
\par 2
\par(1)
    $\left(
    \begin{array}{cccc}
        1 & 1 & \vdots & 0 \\
        2 & 1 & \vdots & 4
    \end{array}
    \right) \to$  $\left(
    \begin{array}{cccc}
        1 & 1 & \vdots & 0 \\
        1 & 0 & \vdots & 4 \\
    \end{array}
    \right) \to $  $
\left(
\begin{array}{cccc}
    0 & 1 & -4 \\
    1 & 0 & 4
\end{array}
\right)
$\\
    よって$x = 4,y = 4$。\\
\par(2)
    $\left(
    \begin{array}{cccccc}
        1 & 2 & 2 & 1 & \vdots & 1 \\
        -1 & 4 & 1 & 1 & \vdots & 2 \\
        2 & 1 & 3 & 1 & \vdots & -1
    \end{array}
    \right) \to  $
    $\left(
    \begin{array}{cccccc}
        1 & 0 & 0 & 0 & \vdots & 0 \\
        0 & 4 & 1 & 1 & \vdots & 2 \\
        0 & 1 & 3 & 1 & \vdots & -1 \\
        0 & 2 & 2 & 1 & \vdots & 1
    \end{array}
    \right) \to  $
    $\left(
    \begin{array}{cccccc}
        1 & 0 & 0 & 0 & \vdots & 0 \\
        0 & 1 & 0 & 0 & \vdots & -1 \\
        0 & 0 & 4 & 1 & \vdots & -3 \\
        0 & 1 & -1 & 0 & \vdots & 2
    \end{array}
    \right)$\\
    $ \to  \left(
    \begin{array}{cccccc}
        1 & 0 & 0 & 0 & \vdots & 0 \\
        0 & 1 & 0 & 0 & \vdots & -1 \\
        0 & 0 & 4 & 1 & \vdots & -3 \\
        0 & 1 & -1 & 0 & \vdots & 2
    \end{array}
    \right) \to  $
    $\left(
    \begin{array}{cccccc}
        1 & 0 & 0 & 0 & \vdots & 0 \\
        0 & 1 & 0 & 0 & \vdots & -1 \\
        0 & 0 & 0 & 1 & \vdots & 9 \\
        0 & 0 & 1 & 0 & \vdots & -3
    \end{array}
    \right) \to  $
    $\left(
    \begin{array}{cccccc}
        1 & 0 & 0 & 0 & \vdots &  0 \\
        0 & 1 & 0 & 0 & \vdots & -1 \\
        0 & 0 & 1 & 0 & \vdots & -3 \\
        0 & 0 & 1 & 0 & \vdots & 9
    \end{array}
    \right)$\\
    よって解は$x = 0,y = -1,z = -3,w = 9$。\\
\par 3
\par(1)(左辺) = $\left(
\begin{array}{cccc}
    x_1+x_2+x_3+x_4 & x_1-x_2+x_3-x_4 & x_1+x_2-x_3-x_ 4 & x_1-x_2-x_3+x_4 \\
    x_1+x_2+x_3+x_4 & -x_1+x_2-x_3+x_4 & x_1+x_2-x_3-x_4 & -x_1+x_2+x_3-x_4 \\
    x_1+x_2+x_3+x_4 & x_1-x_2+x_3-x_4 & -x_1-x_2+x_3+x_4 & -x_1+x_2+x_3-x_4 \\
    x_1+x_2+x_3+x_4 & -x_1+x_2-x_3+x_4 & -x_1-x_2+x_3+x_4 & x_1-x_2-x_3+x_4
\end{array}
\right)$。
(右辺) = \\
$\left(
\begin{array}{cccc}
    x_1+x_2+x_3+x_4 & x_1-x_2+x_3-x_4 & x_1+x_2-x_3-x_4 & x_1-x_2-x_3+x_4 \\
    x_1+x_2+x_3+x_4 & (x_1-x_2+x_3-x_4) & -(x_1+x_2-x_3-x_4) & -(x_1-x_2-x_3+x_4) \\
    x_1+x_2+x_3+x_4 & x_1-x_2+x_3-x_4 & -(x_1+x_2-x_3-x_4) & -(x_1-x_2-x_3+x_4) \\
    x_1+x_2+x_3+x_4 & -(x_1-x_2+x_3-x_4) & -(x_1+x_2-x_3-x_4) & x_1-x_2-x_3+x_4
\end{array}
\right) = $
    $\left(
\begin{array}{cccc}
    x_1+x_2+x_3+x_4 & x_1-x_2+x_3-x_4 & x_1+x_2-x_3-x_ 4 & x_1-x_2-x_3+x_4 \\
    x_1+x_2+x_3+x_4 & -x_1+x_2-x_3+x_4 & x_1+x_2-x_3-x_4 & -x_1+x_2+x_3-x_4 \\
    x_1+x_2+x_3+x_4 & x_1-x_2+x_3-x_4 & -x_1-x_2+x_3+x_4 & -x_1+x_2+x_3-x_4 \\
    x_1+x_2+x_3+x_4 & -x_1+x_2-x_3+x_4 & -x_1-x_2+x_3+x_4 & x_1-x_2-x_3+x_4
\end{array}
\right)$となり、確かにこの等式は成り立つ。\\
(2)(与式) = $x_1^4+{x_2}^2{x_4}+x_3^4+{x_2}^2{x_4}^2 -x_4^4 -{x_1}^2{x_3}^2-x_2^4-{x_1}^2{x_3}^2 = (x_1^2 - x_3^2)^2-(x_2^2 - x_4^2)^2$。\\
\par 4
\par n = 2のとき$
\left[
\begin{array}{cc}
    1 & x_1 \\
    1 & x_2
\end{array}
\right] = x_2 - x_1$となって成立。\\
n = kのとき成立するとする。n = k + 1のときk + 1行から$x_1 \times$第k行を引き、第k行から$x_1 \times$第k - 1行を引く$...$という操作を繰り返すと、\\
    $\left[
    \begin{array}{ccccc}
        1 & 1 & 1 & \cdots & 1 \\
        0 & x_2-x_1 & x_3-x_1 & \cdots & x_{k+1}-x_1 \\
        0 & x_2(x_2-x_1) & x_3(x_3-x_1) & \cdots & x_{k+1}(x_{k+1}-x_1) \\
        \vdots & \vdots & \vdots & \ddots & \vdots \\
        0 & x_2^{k-1}(x_2-x_1) &x_3^{k-1}(x_3-x_1) \vdots & x_{k+1}^{k-1}(x_{k+1}-x_1)
    \end{array}
    \right]
    $。\\これを第一列に関して余因子展開すると
    $\left[ \begin{array}{cccc}
                x_2-x_1 & x_3-x_1 & \cdots & x_{k+1}-x_1 \\
                x_2(x_2-x_1) & x_3(x_3-x_1) & \cdots & x_{k+1}(x_{k+1} -x_1) \\
                \vdots & \vdots & \ddots & \vdots \\
                x_2^{k-1}(x_2 -x_1) & x_3^{k-1}(x_3-x_1) & \cdots & x\{k+1^{k-1}(x{k+1} -x_1)
    \end{array} \right]$
$= (x_2 -x_1)(x_3-x_1) \cdots (x_{k+1}-x_1)\left[
\begin{array}{cccc}
    1 & 1 & \cdots & 1 \\
    x_2 & x_3 & \cdots & x_{k+1} \\
    \vdots & \vdots & \ddots & \vdots \\
    x_2^{k-1} & x_3^{k-1} & \cdots & x_{k+1}^{k-1}
\end{array}
\right]$\\
    $ = (x_2-x_1)(x_3-x_1)\cdots (x_{k+1} -x_1) \prod_{2\leq i < j \leq n}(x_j-x_i)$。
    帰納法の仮定より、1からkまでのすべての組で成り立つことがわかっているので$\prod_{1\leq i < j \leq n}$が成り立つ。\\
 \par5
 \par n個の内のある一点の座標を代入した式を$y_i = a_0 + a_1 x_i + \cdots a_{n-2}x_i^{n-2}+a_{n-1}x_i^{n-1}$とおく。
    これをすべての点に行って、n本の式をまとめて行列にして書くと、xについてまとめた行列は問4のヴァンデルモンド行列の転置である。
    ゆえにxについてまとめた行列の行列式が0でないとき、逆行列が存在して各項の係数が一意に定まる。また、条件より、各$x_i$は異なり、また転置の行列式はもとの行列式と値が等しい。
    ヴァンデルモンド行列の行列式は問4でみたように各$x_i$のうちお互いに等しいものが存在するとき、またその時に限って行列式の値が0になるので今回は0以外の
    値を取り、このとき係数$a_j(j = 0,1,\cdots n-1)$はそれぞれ一意に定まる。よって題意は証明された。

\end{document}