%! Author = chicken1925
%! Date = 2020/12/04

% Preamble
\documentclass[11pt]{jsarticle}

% Packages
\usepackage{amsmath}
\usepackage{amsfonts}
% Document
\begin{document}
2119116s 佐野 海徳\\
%HW41
%par $\sqrt[4]{2} = a + b\sqrt{2}, (a,b \in \mathbb{Q})$とおく。両辺を2乗すると、$\sqrt{2} = a^2 + 2b^2 + 2ab\sqrt{2}$となるので$\sqrt[4]{2}$はM上2次。また、M上1次の元はすべてQ上2次である。
%よって、
HW42
\par$f(x) = 4x^3 - 3x + \frac{1}{2}$とする$f(x + 1) = \frac{1}{2}{8(x + 1)^3 -6(x + 1) + 1} = \frac{1}{2}(8x^3 + 24x^2-18x + 3)$となる。ここで、$g(x) = 8x^3 + 24x^2 + 18x + 3$が既約であることが証明できればよい。
$p = 3$とすると、3はpの倍数であり、$p^2$の倍数でない。また、$x^2,x$の係数である$24,18$はともにpの倍数である。さらに$x^3$の係数である8はpの倍数でないのでアイゼンシュタインの既約判定法よりf(x)は既約である。
\end{document}