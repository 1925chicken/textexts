%! Author = chicken1925
%! Date = 2020/12/09

% Preamble
\documentclass[11pt]{jsarticle}

% Packages
\usepackage{amsmath}
\usepackage{amsfonts}

% Document
\begin{document}
2119116s 佐野海徳\\
HW43
\par $\mathbb{Q}(\sqrt{2})$は2次であり、$\mathbb{Q}(\sqrt{3})$も2次。よって、$\mathbb{Q}(\sqrt{2},\sqrt{3})$は連鎖律より4次以下。一方再び連鎖律より2の倍数であるので2の倍数である。よって$\mathbb{Q}(\sqrt{2},\sqrt{3})$の次数は2である。
\\HW44
\par 生成多項式を$x^3 + x + 1$とする。このときの0以外の元の積の逆元を求める。まずは1について求めるとこの体では$1 \times 1 = 1$である。よって1の逆元は自分自身である。xについて求めると、$x \times (1 + x^2) = 1$であるから、xの逆元は$1 + x^2$である。
同様にすると、$1 + x$の逆元は$x + x^2$,$x^2$の逆元は$1 + x + x^2$、$1 + x^2$の逆元は$x$,$x + x^2$の逆元は$1 + x$,$1 + x + x^2$の逆元は$x^2$である。


\end{document}