%! Author = chicken1925
%! Date = 2020/10/24

% Preamble
\documentclass[11pt]{jsarticle}

% Packages
\usepackage{amsmath}

% Document
\begin{document}
2119116s 佐野 海徳\\
HW23\\
$\{1,-1,i,-i\}$を$\{i,-i\}$でコセット分解すると$\{1.-1\}\cup\{i,-i\}$となる。ここで、$\{i,-i\}$には生成元iが含まれており、$\{1,-1\}$には単位元1が含まれている。
このことから、$C_2$に対応する全単射は$0 \mapsto \{1,-1\}, 1 \mapsto \{i,-i\}$となるから同型となる群が与えられた。\\
HW24\\元がのように互換の積で表されるかを考える。例えば(1234)なら(14)(13)(12)、(14)(23)はそのまま、(123)なら(13)(12)である。
ここで(123)と(13)(24)は同じ文字をすべての置換に含む互換かそうでないかで異なる。よってeを含む共役類の元はeのみの1個、(12)を含む共役類の元は(12),(13),
(14),(23),(24),(34)の6つ、(123)と同じ共役類の元は(123),(124),(132),(142),(134),(143),(234),(243)の8つ、(12)(34)と同じ共役類の元は(12)(34),(13)(24),(14)(23)の3つ、
(1234)と同じ共役類の元は(1234),(1243),(1324),(1342),(1423),(1432)の6つ。よって類等式は24 = 1 + 6 + 8 + 3 +  6となり、確かに右辺に出現する数は24の約数である。\\

\end{document}